%% Custom commands
%% ===============

%%%% Fancy comments macros ====================================
\newif\ifremovecomments
%% Change the following to hide all comments (accept all suggestions, deletions and replacements)
\removecommentsfalse
%\removecommentstrue

\makeatletter
\newcommand{\mysubscript}[2]{\textsubscript{\textcolor{#2}{\textsf{\textbf{#1}}}}}
\newcommand*\defcomment[4]{% 1. Long name, 2. short name, 3. initials, 4. color
  \ifremovecomments
    \expandafter\newcommand\csname #1\endcsname[1]{%
    }
    \expandafter\newcommand\csname @#2delnoname\endcsname[1]{%
    }
    \expandafter\newcommand\csname #2del\endcsname[1]{%
    }
    \expandafter\newcommand\csname #2sugg\endcsname[1]{##1}
    \expandafter\newcommand\csname #2subs\endcsname[2]{##2}
  \else
    \expandafter\newcommand\csname #1\endcsname[1]{%
        \textcolor{#4}{\ding{110}\mysubscript{#3}{#4}\,{##1}}%
    }
    \expandafter\newcommand\csname @#2delnoname\endcsname[1]{%
        \texorpdfstring%
        {\bgroup\markoverwith{\textcolor{#4}{\rule[0.35ex]{2pt}{1pt}}}\ULon{##1}}%
        {##1}%
    }
    \expandafter\MakeRobust\csname @#2delnoname\endcsname
    \expandafter\newcommand\csname #2del\endcsname[1]{%
        \csname @#2delnoname\endcsname{##1}\kern0.1em\mysubscript{#3}{#4}%
    }
    \expandafter\newcommand\csname #2sugg\endcsname[1]{%
        \textcolor{#4}{[##1]\mysubscript{#3}{#4}}%
    }
    \expandafter\newcommand\csname #2subs\endcsname[2]{%
        \csname @#2delnoname\endcsname{##1}\csname #2sugg\endcsname{##2}%
    }
  \fi
    \expandafter\newcommand\csname #2sout\endcsname{\csname #2del\endcsname}
}
%% Daniele: see https://tex.stackexchange.com/a/173245
\makeatother

%%%%% Define new users for comments here ======================
\defcomment{daniele}{da}{DA}{magenta}

%% Daniele: old comment macros (no fanciness)
%\newcommand{\danielecolor}{magenta}
%\newcommand{\danielesubscript}{\textsubscript{\textcolor{\danielecolor}{\textsf{\textbf{DA}}}}}
%\newcommand{\daniele}[1]{\textcolor{\danielecolor}{\ding{110}\danielesubscript~{#1}}}
%\newcommand{\dadelnoname}[1]{\bgroup\markoverwith{\textcolor{\danielecolor}{\rule[0.35ex]{2pt}{1pt}}}\ULon{#1}}
%\newcommand{\dadel}[1]{\dadelnoname{#1}\kern0.1em\danielesubscript}
%\newcommand{\dasout}{\dadel}
%\newcommand{\dasugg}[1]{\textcolor{\danielecolor}{[#1]\danielesubscript}}
%\newcommand{\dasubs}[2]{\dadelnoname{#1}\dasugg{#2}}


%% General macros for style and similar
%% ===========================

%% Change to paragraph style
\renewcommand\paragraph[1]{\smallskip\noindent\textbf{\textsf{#1.\,}}}

%% Special emphasis in bold - use only for very strong emphasis
\newcommand{\strongemph}[1]{\textbf{#1}}

%% Frequently used standard abbreviations (i.e., e.g., etc.)
\newcommand{\etc}{etc.\@\xspace}
\newcommand{\ie}{i.e.,\@\xspace}
\newcommand{\eg}{e.g.,\@\xspace}
\newcommand{\cf}{cf.\@\xspace}
\newcommand{\etal}{et~al.\@\xspace}
%\newcommand{\Dr}{Dr.\@~}
%\newcommand{\Prof}{Prof.\@~}
%\newcommand{\ProfDr}{Prof.\@~Dr.\@~}


%% Tweaks for floats (makes them behave better) -- provided by Adrian
%% See also http://aty.sdsu.edu/~aty/bibliog/latex/floats.html
\renewcommand{\textfraction}{0}
\renewcommand{\topfraction}{1}
\renewcommand{\bottomfraction}{1}
\setcounter{totalnumber}{10}
\setcounter{topnumber}{10}
\setcounter{bottomnumber}{10}
\setcounter{dbltopnumber}{10}
\renewcommand{\floatpagefraction}{1}
\renewcommand{\dblfloatpagefraction}{1}


%% Spacing tricks for sections
%\usepackage{titlesec}
%\titlespacing*{\section}{0pt}{2.5ex}{1.0ex}
%\titlespacing*{\subsection}{0pt}{2ex}{1.0ex}

%\setlength{\belowcaptionskip}{-5pt} % Figures and tables

%\setlist[itemize]{leftmargin=1em, itemsep=-0.1em, itemindent=0em, topsep=1pt}
%\setlist[enumerate]{leftmargin=1.2em, itemindent=0em, itemsep=5pt, topsep=1pt}

%\let\OLDthebibliography\thebibliography
%\renewcommand\thebibliography[1]{
%   \OLDthebibliography{#1}
%   \setlength{\parskip}{0pt}
%   \setlength{\itemsep}{0pt plus 0.3ex}
%   %raggedright
%}


%% Macros for the protocol
%% ==================

%% Insert protocol specific macros in the area below
%% -------------------------------------------------------------------------------------


%% -------------------------------------------------------------------------------------

%% Macros for math and crypto
\newcommand{\concat}{\ensuremath{\mathbin{\|}}}
\newcommand{\xor}{\oplus}
\newcommand{\enc}{\ensuremath{\mathsf{enc}}}
\newcommand{\dec}{\ensuremath{\mathsf{dec}}}
\newcommand{\mac}{\ensuremath{\mathsf{MAC}}}
\newcommand{\prp}{\ensuremath{\mathsf{PRP}}}
\newcommand{\prg}{\ensuremath{\mathsf{PRG}}}

% Other math utilities
%\newcommand{\concat}{\ensuremath{\mathbin{\|}}}

%% Macros for algorithms
% Statex with indent (for algorithms)
\makeatletter
\newcommand{\StatexIndent}[1][3]{%
    \setlength\@tempdima{\algorithmicindent}%
    \Statex\hskip\dimexpr#1\@tempdima\relax}
\makeatother
\renewcommand{\algorithmicrequire}{\textbf{Input:\ \ \;}}
\renewcommand{\algorithmicensure}{\textbf{Output:}}

%% Special characters for number sets, e.g. real or complex numbers.
%\newcommand{\C}{\mathbb{C}}
%\newcommand{\K}{\mathbb{K}}
%\newcommand{\N}{\mathbb{N}}
%\newcommand{\Q}{\mathbb{Q}}
%\newcommand{\R}{\mathbb{R}}
%\newcommand{\Z}{\mathbb{Z}}
%\newcommand{\X}{\mathbb{X}}

%% Fixed/scaling delimiter examples (see mathtools documentation)
%\DeclarePairedDelimiter\abs{\lvert}{\rvert}
%\DeclarePairedDelimiter\norm{\lVert}{\rVert}

%% Use the alternative epsilon per default and define the old one as \oldepsilon
\let\oldepsilon\epsilon
\renewcommand{\epsilon}{\ensuremath\varepsilon}

%% Also set the alternate phi as default.
\let\oldphi\phi
\renewcommand{\phi}{\ensuremath{\varphi}}
