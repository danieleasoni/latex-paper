%% Custom commands
%% ===============

%% Comments
\newcommand{\david}{\ding{43}\textcolor{red} {DB:}\textcolor{magenta} }
\newcommand{\daniele}{\ding{110}\ding{43}\textcolor{red} {Daniele:}\textcolor{red} }
\newcommand{\takayuki}{\ding{110}\ding{43}\textcolor{green} {Daniele:}\textcolor{red} }


%% General macros for style and similar
%% ===========================

%% Change to paragraph style
\renewcommand\paragraph[1]{\smallskip\noindent\textbf{\textsf{#1.\,}}}

%% Special emphasis in bold - use only for very strong emphasis
\newcommand{\strongemph}[1]{\textbf{#1}}

%% Frequently used standard abbreviations (i.e., e.g., etc.)
\newcommand{\etc}{etc.\@\xspace}
\newcommand{\ie}{i.e.,\@\xspace}
\newcommand{\eg}{e.g.,\@\xspace}
\newcommand{\cf}{cf.\@\xspace}
\newcommand{\etal}{et~al.\@\xspace}
%\newcommand{\Dr}{Dr.\@~}
%\newcommand{\Prof}{Prof.\@~}
%\newcommand{\ProfDr}{Prof.\@~Dr.\@~}


%% Tweaks for floats (makes them behave better) -- provided by Adrian
%% See also http://aty.sdsu.edu/~aty/bibliog/latex/floats.html
\renewcommand{\textfraction}{0}
\renewcommand{\topfraction}{1}
\renewcommand{\bottomfraction}{1}
\setcounter{totalnumber}{10}
\setcounter{topnumber}{10}
\setcounter{bottomnumber}{10}
\setcounter{dbltopnumber}{10}
\renewcommand{\floatpagefraction}{1}
\renewcommand{\dblfloatpagefraction}{1}


%% Spacing tricks for sections
%\usepackage{titlesec}
%\titlespacing*{\section}{0pt}{2.5ex}{1.0ex}
%\titlespacing*{\subsection}{0pt}{2ex}{1.0ex}

%\setlength{\belowcaptionskip}{-5pt} % Figures and tables

%\setlist[itemize]{leftmargin=1em, itemsep=-0.1em, itemindent=0em, topsep=1pt}
%\setlist[enumerate]{leftmargin=1.2em, itemindent=0em, itemsep=5pt, topsep=1pt}

%\let\OLDthebibliography\thebibliography
%\renewcommand\thebibliography[1]{
%	\OLDthebibliography{#1}
%	\setlength{\parskip}{0pt}
%	\setlength{\itemsep}{0pt plus 0.3ex}
%	%raggedright
%}


%% Macros for the protocol
%% ==================

%% Insert protocol specific macros in the area below
%% -------------------------------------------------------------------------------------


%% -------------------------------------------------------------------------------------

%% Macros for math and crypto
\newcommand{\concat}{\ensuremath{\mathbin{\|}}}
\newcommand{\xor}{\oplus}
\newcommand{\enc}{\ensuremath{\mathsf{enc}}}
\newcommand{\dec}{\ensuremath{\mathsf{dec}}}
\newcommand{\mac}{\ensuremath{\mathsf{MAC}}}
\newcommand{\prp}{\ensuremath{\mathsf{PRP}}}
\newcommand{\prg}{\ensuremath{\mathsf{PRG}}}

% Other math utilities
%\newcommand{\concat}{\ensuremath{\mathbin{\|}}}

%% Macros for algorithms
% Statex with indent (for algorithms)
\makeatletter
\newcommand{\StatexIndent}[1][3]{%
	\setlength\@tempdima{\algorithmicindent}%
	\Statex\hskip\dimexpr#1\@tempdima\relax}
\makeatother
\renewcommand{\algorithmicrequire}{\textbf{Input:\ \ \;}}
\renewcommand{\algorithmicensure}{\textbf{Output:}}

%% Special characters for number sets, e.g. real or complex numbers.
%\newcommand{\C}{\mathbb{C}}
%\newcommand{\K}{\mathbb{K}}
%\newcommand{\N}{\mathbb{N}}
%\newcommand{\Q}{\mathbb{Q}}
%\newcommand{\R}{\mathbb{R}}
%\newcommand{\Z}{\mathbb{Z}}
%\newcommand{\X}{\mathbb{X}}

%% Fixed/scaling delimiter examples (see mathtools documentation)
%\DeclarePairedDelimiter\abs{\lvert}{\rvert}
%\DeclarePairedDelimiter\norm{\lVert}{\rVert}

%% Use the alternative epsilon per default and define the old one as \oldepsilon
\let\oldepsilon\epsilon
\renewcommand{\epsilon}{\ensuremath\varepsilon}

%% Also set the alternate phi as default.
\let\oldphi\phi
\renewcommand{\phi}{\ensuremath{\varphi}}
